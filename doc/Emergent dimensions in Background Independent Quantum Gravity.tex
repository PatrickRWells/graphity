\documentclass[prd,12pt,nofootinbib]{revtex4}
\usepackage{graphicx}
\usepackage{epstopdf}
\usepackage{setspace}
\usepackage{amsmath}
\usepackage{latexsym, amssymb}
\usepackage{verbatim}
\usepackage{lscape}
\usepackage{rotating}
\usepackage[export]{adjustbox}
\usepackage{subfloat}
%\usepackage{feynmp}
%\DeclareGraphicsRule{*}{mps}{*}{}

%  personal abbreviations and macros

%%%%%%%%%%%%%%%%%%%%%%%%%%%%%%%%%%%%%%%%%%%%%%%%%%%%%%%%%%%%%%%%%%%%
%%  basic formatting macros:
%%%%%%%%%%%%%%%%%%%%%%%%%%%%%%%%%%%%%%%%%%%%%%%%%%%%%%%%%%%%%%%%%%%

%%  single-line equations:

\def\beq{\begin{equation}}
\def\eeq#1{\label{#1}\end{equation}}
\def\eeqn{\end{equation}}

%%  multiple-line equations  (use \CR as the carriage return):

\newenvironment{Eqnarray}%
   {\arraycolsep 0.14em\begin{eqnarray}}{\end{eqnarray}}
\def\beqa{\begin{Eqnarray}}
\def\eeqa#1{\label{#1}\end{Eqnarray}}
\def\eeqan{\end{Eqnarray}}
\def\CR{\nonumber \\ }
\def\m#1{\mathcal{#1}}

%%  reference to an equation number:

\def\leqn#1{(\ref{#1})}

%% enumerate
\def\ben{\begin{enumerate}}
\def\een{\end{enumerate}}

%%%%%%%%%%%%%%%%%%%%%%%%%%%%%%%%%%%%%%%%%%%%%%%%%%%%%%%%%%%%%%%%%%%%%%%%

%%  bibliographic entries   (use this or the SPIRES LaTeX output)

%%   Journal or bibliographic formatting macros are obsolete!
%%   The SLAC/SPIRES database supplies properly formatted citations;
%%      click on  LaTeX(US) or LaTeX(EU)   

%%%%%%%%%%%%%%%%%%%%%%%%%%%%%%%%%%%%%%%%%%%%%%%%%%%%%%%%%%%%%%%%%%%%%%%%%

%%  sizing and bars

\def\st{\scriptstyle}
\def\sst{\scriptscriptstyle}

\def\overbar#1{\overline{#1}}
\let\littlebar=\bar
\let\bar=\overbar

\def\subscr#1{{\mbox{\scriptsize #1}}}


%%%%%%%%%%%%%%%%%%%%%%%%%%%%%%%%%%%%%%%%%%%%%%%%%%%%%%%%%%%%%%%%%%%%%%%%%

%%  text-mode macros:

\def\etal{{\it et al.}}
\def\ie{{\it i.e.}}
\def\eg{{\it e.g.}}

%%%%%%%%%%%%%%%%%%%%%%%%%%%%%%%%%%%%%%%%%%%%%%%%%%%%%%%%%%%%%%%%%%%%%%%%%%

%%  expectation values:

\def\VEV#1{\left\langle{ #1} \right\rangle}
\def\bra#1{\left\langle{ #1} \right|}
\def\ket#1{\left| {#1} \right\rangle}
\def\vev#1{\langle #1 \rangle}
\newcommand{\vvr}[1]{|{#1}\rangle}
\newcommand{\vvl}[1]{\langle{#1}|}
\newcommand{\lng}{\langle}
\newcommand{\rng}{\rangle}

%%%%%%%%%%%%%%%%%%%%%%%%%%%%%%%%%%%%%%%%%%%%%%%%%%%%%%%%%%%%%%%%%%%%%%%%%

%% relation symbols

\def\lsim{\mathrel{\raise.3ex\hbox{$<$\kern-.75em\lower1ex\hbox{$\sim$}}}}
\def\gsim{\mathrel{\raise.3ex\hbox{$>$\kern-.75em\lower1ex\hbox{$\sim$}}}}

\def\Im{{\rm Im}}
\def\Re{{\rm Re}}

%%%%%%%%%%%%%%%%%%%%%%%%%%%%%%%%%%%%%%%%%%%%%%%%%%%%%%%%%%%%%%%%%%%%%%%%%%%%%%
%%  caligraphic letters (for matrix elements, luminosity, etc.)

\def\D{{\cal D}}
\def\L{{\cal L}}
\def\M{{\cal M}}
\def\O{{\cal O}}
\def\W{{\cal W}}

%%%%%%%%%%%%%%%%%%%%%%%%%%%%%%%%%%%%%%%%%%%%%%%%%%%%%%%%%%%%%%%%%%%%%

%%  matrix operations and fractions:

\def\One{{\bf 1}}
\def\hc{{\mbox{\rm h.c.}}}
\def\tr{{\mbox{\rm tr}}}
\def\half{\frac{1}{2}}
\def\thalf{\frac{3}{2}}
\def\third{\frac{1}{3}}
\def\tthird{\frac{2}{3}}
\newcommand{\fr}[2]{\frac{#1}{#2}}
%\def\Over#1#2{\frac{#1}{#2}}
%\def\del{\partial}
\def\p{\partial}
\def\Dslash{\not{\hbox{\kern-4pt $D$}}}
\def\dslash{\not{\hbox{\kern-2pt $\del$}}}
\def\over#1:#2{\frac{#1}{#2}}

%%%%%%%%%%%%%%%%%%%%%%%%%%%%%%%%%%%%%%%%%%%%%%%%%%%%%%%%%%%%%%%%%%%%%%%%%%%%%

%%  high-energy physics terminology:

\def\Pl{{\mbox{\scriptsize Pl}}}
\def\eff{{\mbox{\scriptsize eff}}}
\def\CM{{\mbox{\scriptsize CM}}}
\def\GUT{{\mbox{\scriptsize GUT}}}
\def\BR{\mbox{\rm BR}}
\def\ee{e^+e^-}
\def\sstw{\sin^2\theta_w}
\def\cstw{\cos^2\theta_w}
\def\mz{m_Z}
\def\gz{\Gamma_Z}
\def\mw{m_W}
\def\mt{m_t}
\def\gt{\Gamma_t}
\def\mh{m_h}
\def\gmu{G_\mu}
\def\GF{G_F}
\def\alphas{\alpha_s}
\def\msb{{\bar{\scriptsize M \kern -1pt S}}}
\def\lmsb{\Lambda_{\msb}}
\def\drb{{\bar{\scriptsize D \kern -1pt R}}}
\def\ELER{e^-_Le^+_R}
\def\EREL{e^-_Re^+_L}
\def\ELEL{e^-_Le^+_L}
\def\ERER{e^-_Re^+_R}
\def\eps{\epsilon}

%%%%%%%%%%%%%%%%%%%%%%%%%%%%%%%%%%%%%%%%%%%%%%%%%%%%%%%%%%%%%%%%%%%%%%%%%%%%%

%%  supersymmetry:

\def\ch#1{\widetilde\chi^+_{#1}}
\def\chm#1{\widetilde\chi^-_{#1}}
\def\neu#1{\widetilde\chi^0_{#1}}
\def\s#1{\widetilde{#1}}

%%%%%%%%%%%%%%%%%%%%%%%%%%%%%%%%%%%%%%%%%%%%%%%%%%%%%%%%%%%%%%%%%%%%%%%%%%%%%5

\makeatletter
\def\section{\@startsection{section}{0}{\z@}{5.5ex plus .5ex minus
 1.5ex}{2.3ex plus .2ex}{\large\bf}}
\def\subsection{\@startsection{subsection}{1}{\z@}{3.5ex plus .5ex minus
 1.5ex}{1.3ex plus .2ex}{\normalsize\bf}}
\def\subsubsection{\@startsection{subsubsection}{2}{\z@}{-3.5ex plus
-1ex minus  -.2ex}{2.3ex plus .2ex}{\normalsize\sl}}

%%%%%%%%%%%%%%%%%%%%%%%%%%%%%%%%%%%%%%%%%%%
% small size table and figure captions %
%%%%%%%%%%%%%%%%%%%%%%%%%%%%%%%%%%%%%%%%%%%
\renewcommand{\@makecaption}[2]{%
   \vskip 10pt
   \setbox\@tempboxa\hbox{\small #1: #2}
   \ifdim \wd\@tempboxa >\hsize     % IF longer than one line:
       \small #1: #2\par          %   THEN set as ordinary paragraph.
     \else                        %   ELSE  center.
       \hbox to\hsize{\hfil\box\@tempboxa\hfil}
   \fi}

%%%%%%%%%%%%%%%%%%%%%%%%%%%%%%%%%%%%%%%%%%%%%%%%%%%%%%%%%%%%%%%%%%%%
% macros to collapse citation numbers to ranges %
%%%%%%%%%%%%%%%%%%%%%%%%%%%%%%%%%%%%%%%%%%%%%%%%%%%%%%%%%%%%%%%%%%%%
% \citenum emits the plain citation number without ornament
% \citea puts it's argument into the ornamentation for citations
% thus \cite{foo} is equivalent to \citea{\citenum{foo}}
 \def\citenum#1{{\def\@cite##1##2{##1}\cite{#1}}}
%\def\citea#1{\@cite{#1}{}}
 
% Collapse citation numbers to ranges.  Non-numeric and undefined labels
% are handled.  No sorting is done.  E.g., 1,3,2,3,4,5,foo,1,2,3,?,4,5
% gives 1,3,2-5,foo,1-3,?,4,5
\newcount\@tempcntc
\def\@citex[#1]#2{\if@filesw\immediate\write\@auxout{\string\citation{#2}}\fi
  \@tempcnta\z@\@tempcntb\m@ne\def\@citea{}\@cite{\@for\@citeb:=#2\do
    {\@ifundefined
       {b@\@citeb}{\@citeo\@tempcntb\m@ne\@citea\def\@citea{,}{\bf ?}\@warning
       {Citation `\@citeb' on page \thepage \space undefined}}%
    {\setbox\z@\hbox{\global\@tempcntc0\csname b@\@citeb\endcsname\relax}%
     \ifnum\@tempcntc=\z@ \@citeo\@tempcntb\m@ne
       \@citea\def\@citea{,}\hbox{\csname b@\@citeb\endcsname}%
     \else
      \advance\@tempcntb\@ne
      \ifnum\@tempcntb=\@tempcntc
      \else\advance\@tempcntb\m@ne\@citeo
      \@tempcnta\@tempcntc\@tempcntb\@tempcntc\fi\fi}}\@citeo}{#1}}
\def\@citeo{\ifnum\@tempcnta>\@tempcntb\else\@citea\def\@citea{,}%
  \ifnum\@tempcnta=\@tempcntb\the\@tempcnta\else
  {\advance\@tempcnta\@ne\ifnum\@tempcnta=\@tempcntb \else\def\@citea{--}\fi
    \advance\@tempcnta\m@ne\the\@tempcnta\@citea\the\@tempcntb}\fi\fi}
%%%%%%%%%%%%%%%%%%%%%%%%%%%%%%%%%%%%%%%%%%%%%%%%%%%%%%%%%%%%%%%%%%%%%%%%
\makeatother



%%%%%%%%%%%%%%%%%%%%%%%%%%%%%%%%%%%%%%%%%%%%%%%%%%%%%%%%%%%%%%%%%%%%
% basic data for the eprint:
%%%%%%%%%%%%%%%%%%%%%%%%%%%%%%%%%%%%%%%%%%%%%%%%%%%%%%%%%%%%%%%%%%%%

\textwidth=6.0in  \textheight=8.25in

%%  Adjust these for your printer:
\oddsidemargin=+0.3in   \topmargin=-0.20in
\parskip=0.1truein

%% preprint number data:
\newcommand\pubnumber{SLAC--PUB--16103}
\newcommand\pubdate{\today}

%%  address and funding acknowledgement data:

\def\SLAC{SLAC National Accelerator Laboratory,\\ 2575 Sand Hill Road, Menlo Park, CA 94025, USA}
\def\doeack{\footnote{Work supported by the US Department of Energy,
                     contract DE--AC02--76SF00515.}}

%%%%%%%%%%%%%%%%%%%%%%%%%%%%%%%%%%%%%%%%%%%%%%%%%%%%%%%%%%%%%%%%%%%%%%%%%%%%
%   document style macros
%%%%%%%%%%%%%%%%%%%%%%%%%%%%%%%%%%%%%%%%%%%%%%%%%%%%%%%%%%%%%%%%%%%%%%%%%%%%
\def\Title#1{\begin{center} {\Large #1 } \end{center}}
\def\Author#1{\begin{center}{ \sc #1} \end{center}}
\def\Address#1{\begin{center}{ \it #1} \end{center}}
\def\andauth{\begin{center}{and} \end{center}}
\def\submit#1{\begin{center}Submitted to {\sl #1} \end{center}}
\newcommand\pubblock{\rightline{\begin{tabular}{l} \pubnumber\\
         \pubdate \end{tabular}}}
\newenvironment{Abstract}{\begin{quotation} \begin{center}
                       ABSTRACT
     \end{center}\bigskip  }{\end{quotation}}
\newenvironment{Presented}{\begin{quotation} \begin{center} 
             PRESENTED AT\end{center}\bigskip 
      \begin{center}\begin{large}}{\end{large}\end{center} \end{quotation}}
\def\submit#1{\begin{center}Submitted to {\sl #1} \end{center}}
\def\Acknowledgements{\bigskip  \bigskip \begin{center} \begin{large}
             \bf ACKNOWLEDGEMENTS \end{large}\end{center}}
%%%%%%%%%%%%%%%%%%%%%%%%%%%%%%%%%%%%%%%%%%%%%%%%%%%%%%%%%%%%%%%%%%%%%%%%%%%%
%  personal abbreviations and macros

%\input mymacros.tex
\def\spa#1#2{\langle #1 #2 \rangle}
\def\spb#1#2{[ #1 #2 ]}
\def\Mc{{\bf M}}
\def\smin{s_{\mbox{\scriptsize min}}}
\def\sminc{s_{\mbox{\scriptsize min,C}}}
\def\soutc{s_{{\mbox{\scriptsize out}},C}}
\def\sab{s_{ab}}
\def\tout{t_{\mbox{\scriptsize out}}}
\def\toutc{t_{{\mbox{\scriptsize out}},C}}
\def\minmsq{m^2_{\mbox{\scriptsize min}}}
\def\qq{q\bar{q}}
\def\min{{\mbox{\scriptsize min}}}
\def\max{{\mbox{\scriptsize max}}}


%%%%%%%%%%%%%%%%%%%%%%%%%%%%%%%%%%%%%%%%%%%%%%%%%%%%%%%%%%%%%%%%%%%%%%%%%%%



%%%%%%%%%%%%%%%%%%%%%%%%%%%%%%%%%%%%%%%%%%%%%%%%%%%%%%%%%%%%%%%%%%%%
% basic data for the eprint:
%%%%%%%%%%%%%%%%%%%%%%%%%%%%%%%%%%%%%%%%%%%%%%%%%%%%%%%%%%%%%%%%%%%%

\textwidth=6.0in  \textheight=8.25in

%%  Adjust these for your printer:
\oddsidemargin=+0.3in   \topmargin=-0.20in
\parskip=0.1truein

%% preprint number data:
%\newcommand\pubnumber{SLAC--PUB--16103}
%\newcommand\pubdate{\today}

%%  address and funding acknowledgement data:

%\def\Stanford{Pepperdine University,\\ 24255 Pacific Coast Hwy, Malibu, CA 90263, USA}
%\def\SLAC{SLAC National Accelerator Laboratory,\\ 2575 Sand Hill Road, Menlo Park, CA 94025, USA}
%\def\doeack{\footnote{Work supported by the US Department of Energy,
%                     contract DE--AC02--76SF00515.}}


\begin{document}

\begin{titlepage}
\title{Emergent Dimensions in Background Independent Quantum Gravity}
\author{Kassahun Betre$^{\dagger}$, Patrick Wells$^{\dagger}$}
\address{$^{\dagger}$Pepperdine University, 24255 Pacific Coast Hwy, Malibu, CA 90263, USA}

\maketitle

\begin{Abstract}
Abstract goes here.

\end{Abstract}

\end{titlepage}

\section{Introduction}
Background-independent models of spacetime geometry are increasingly common in the search for a quantum theory of gravity. If General Relativity is only an effective theory,  

Introduction \cite{Konopka:2008hp}, \cite{Chen:2012ui}, \cite{Konopka:2008ds}.
\section{Section 1}
\subsection{Evolution of the Model} 
Given an undirected, loopless graph $G$ with vertices $V(G) = \{ 1...N\}$ and edges $E(G)$, a Hamiltonian $H$ can be defined on the graph as follows:
\begin{equation}
H = J\sum_{i,j\in V(G), i\neq j}(d_i - d_j)^2 + K\sum_{v\in V(G)}d_v
\end{equation}
Where $J$ and $K$ are weighting constants, and $d_i$ simply represents the degree (number of connected edges) of a given node. The first of the sum forces the graph to be regular in the low-temperature limit, while the second sum pushes the graph to be more or less connected depending on the value of $K$. As $K$ goes to infinity, the expectation value of the average node degree drops to zero. Conversely, as $K$ tends towards negative infinity, the expectation value of the average node degree should go to $N-1$.
A set of seed graphs are evolved according to this Hamiltonian using a Monte-Carlo simulation with a Metropolis algorithm. 
\section{Geometrical Properties of Graphs}
Geometrical properties of the graphs considered are defined as in \cite{Knill:2011}. This definition of dimensionality is computationally problematic, not due to the complexity of the algorithm but rather due to its fundamentally recursive nature. Calculating the dimensionality of a complete graph $K_N$  on $N$ nodes requires $N!$ recursive calls, which quickly becomes computationally intractable as N increases. Of course, $K_N$ is just an N-1-dimensional simplex, but no such simplification exists for highly dense graphs close to $K_N$. While the graphs considered in the simulation were in general not so dense as to require such a high number of recursive calls, they were still complex enough to impose an upper limit on the size of the graphs considered.

The Euler Characteristic $\chi$ is equal to the sum of the curvatures at every node, as defined in \cite{Knill:2011}
\section{Conclusion}  
\begin{thebibliography}{99}
  


%\cite{Konopka:2008hp}
\bibitem{Konopka:2008hp} 
  T.~Konopka, F.~Markopoulou and S.~Severini,
  %``Quantum Graphity: A Model of emergent locality,''
  Phys.\ Rev.\ D {\bf 77}, 104029 (2008)
  doi:10.1103/PhysRevD.77.104029
  [arXiv:0801.0861 [hep-th]].
  %%CITATION = doi:10.1103/PhysRevD.77.104029;%%
  %92 citations counted in INSPIRE as of 24 May 2016

%\cite{Chen:2012ui}
\bibitem{Chen:2012ui} 
  S.~Chen and S.~S.~Plotkin,
  %``Statistical mechanics of graph models and their implications for emergent spacetime manifolds,''\cite{Chen:2012ui}
  Phys.\ Rev.\ D {\bf 87}, no. 8, 084011 (2013)
  doi:10.1103/PhysRevD.87.084011
  [arXiv:1210.3372 [gr-qc]].
  %%CITATION = doi:10.1103/PhysRevD.87.084011;%%
  %2 citations counted in INSPIRE as of 25 Aug 2016


%\cite{Konopka:2008ds}
\bibitem{Konopka:2008ds} 
  T.~Konopka,
  %``Statistical Mechanics of Graphity Models,''
  Phys.\ Rev.\ D {\bf 78}, 044032 (2008)
  doi:10.1103/PhysRevD.78.044032
  [arXiv:0805.2283 [hep-th]].
  %%CITATION = doi:10.1103/PhysRevD.78.044032;%%
  %24 citations counted in INSPIRE as of 24 Jun 2016

%\cite{Conrady:2010qz}
\bibitem{Conrady:2010qz} 
  F.~Conrady,
  %``Space as a low-temperature regime of graphs,''
  J.\ Statist.\ Phys.\  {\bf 142}, 898 (2011)
  doi:10.1007/s10955-011-0135-9
  [arXiv:1009.3195 [gr-qc]].
  %%CITATION = doi:10.1007/s10955-011-0135-9;%%
  %7 citations counted in INSPIRE as of 09 May 2018
  
  \bibitem{Knill:2011}
   O.~Knill
   %``On the Dimensionality and Euler Characteristic of Random Graphs``
   [arXiv:1112.5749 [math.PR]]
  
  \bibitem{nautyTraces}
  McKay, B.D. and Piperno, A.,
  Practical Graph Isomorphism, II, 
  Journal of Symbolic Computation, 60 (2014), pp. 94-112, http://dx.doi.org/10.1016/j.jsc.2013.09.003
 
\end{thebibliography}




\end{document}
